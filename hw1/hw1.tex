\documentclass{article} % For LaTeX2e
\usepackage{nips14submit_e,times}
\usepackage{hyperref}
\usepackage{url}

% For figures
\usepackage{graphicx} % more modern
\usepackage{pstool}
\usepackage{caption}
\usepackage{subcaption}

% For citations
\usepackage{natbib}

% Nice colors
\usepackage{color,colortbl}

% Table related packages
\usepackage{multirow}
\usepackage{rotating}
% \newcommand\VRule[1][\arrayrulewidth]{\vrule width #1}

% Multiple columns
\usepackage{multicol}

% Enumerate items
\usepackage{enumitem}

% Math packages
\usepackage{amsmath,amsfonts,amssymb,amsthm,dsfont}
\usepackage{bm}
\usepackage{array,booktabs,xcolor}
\usepackage{color}

% Theorem environments
\newtheorem{theorem}{Theorem}
\newtheorem{definition}[theorem]{Definition}
\newtheorem{lemma}[theorem]{Lemma}
\newtheorem{proposition}[theorem]{Proposition}
\newtheorem{corollary}[theorem]{Corollary}

% Colors
\definecolor{LightGreen}{rgb}{0.87,0.90,0.82}
\definecolor{DarkGreen}{rgb}{0.60,0.73,0.35}
\renewcommand*\arraystretch{1.1}

% Proof sketch environment
\newenvironment{proofsketch}{\trivlist\item[]\emph{Proof Sketch}:}%
{\unskip\nobreak\hskip 1em plus 1fil\nobreak$\Box$
\parfillskip=0pt%
\endtrivlist}

\title{CS 267 Homework 1: Matrix Multiplication}

\author{
Karthik S. Narayan \\
Electrical Engineering and Computer Science\\
University of California, Berkeley\\
Berkeley, CA 94709 \\
\texttt{karthik.narayan@berkeley.edu} \\
\And
Charles Scudiere \\
Mechanical Engineering\\
University of California, Berkeley\\
Berkeley, CA 94709 \\
\texttt{cascudiere@berkeley.edu} \\
}

% The \author macro works with any number of authors. There are two commands
% used to separate the names and addresses of multiple authors: \And and \AND.
%
% Using \And between authors leaves it to \LaTeX{} to determine where to break
% the lines. Using \AND forces a linebreak at that point. So, if \LaTeX{}
% puts 3 of 4 authors names on the first line, and the last on the second
% line, try using \AND instead of \And before the third author name.

\newcommand{\fix}{\marginpar{FIX}}
\newcommand{\new}{\marginpar{NEW}}

\nipsfinalcopy % Uncomment for camera-ready version

\begin{document}

% Commonly used expressions
\newcommand{\dgemm}{\texttt{dgemm}}

% TODO commands
\newcommand\todo{\textcolor{red}{TODO}}
\newcommand\tocite{\textcolor{red}{CITE}}

\maketitle

\section{Optimizations Used and Attempted}
This section describes the approach we took in optimizing the \dgemm\; procedure,
which runs at an average $71\%$ of peak speed on an AMD MagnyCours under the
given test cases.
Concretely, we describe how we compute $C \leftarrow C + AB$. At a high level,
we (a) create auxiliary data structures that re-arrange and/or pad $A$, $B$,
and $C$, (b) employ blocking to multiply small blocks of $A$, $B$, and $C$,
where we (c) use SSE instructions with loop unrolling.

\subsection{Inner vs. Outer products}
\label{sec:ioproducts}
Originally, we coded up a standard ``inner-product-based'' matrix multiplication
algorithm. Unfortunately, we could not figure out a good way to perform
horizontal adds using SSE2, i.e. \texttt{\_mm\_hadd\_pd}. As such, we opted to go
for an ``outer-product-based'' method, noting that:
\begin{align}
  \label{eq:matmul}
  C = AB = \sum_{i}\sum_{j}A_i B^{(j)},
\end{align}
where $A_i$ denotes the $i$th row of $A$ and $B^{(j)}$ denotes the $j$th row of
$B$. In particular, the inner summand has the same size as $C$. To evaluate the
inner summand, we consider $4$ entries of $A_i$ and $4$ entries of $B^{(j)}$ at
a time, yielding a $4\times 4$ matrix that we can add to $C$. As an example,
iterating until $A_i$ and $B^{(j)}$ are covered yields a single summand
$A_i B^{(j)}$\footnote{This is just an example for clarity -- in reality, we do
  not do this, but rather iterate over blocks of A.}.

We load in the current values of the $4\times 4$ submatrix of $C$ that we
plan on updating; in particular, we use $8$ XMM registers for this.
In computing each $4\times 4$ contribution, we use $2$ XMM registers to store
the $4$ entries of $A_i$ exactly once. To easily broadcast $B^{(j)}$, we use
$4$ XMM registers for the $4$ entries of $B^{(j)}$, where each register contains
two of the same entry using the \texttt{\_mm\_load1\_pd} intrinsic. We compute
and add the associated multiplies to each of C's $8$ XMM registers. In total,
this procedure uses $8 + 4 + 2 = 14$ XMM registers, just under the total of $16$
provided with the AMD MagnyCours architecture.

To ensure that we work only with aligned SSE instructions and avoid fringe
cases, we pad $A$ and $C$ with zeros ahead of time; specifically, we pad
$A$ and $C$ with zeros so that the rows and columns are each multiples of $4$.
We do not pad $B$, since the \texttt{\_mm\_load1\_pd} intrinsic is unaligned.

\textbf{Testing Unrolling Sizes} In addition to working with $4\times 4$ matrices, we
experimented with $2\times 2$, $6\times 6$, and $8\times 8$ matrices.
Ultimately, $4\times 4$ worked the best, presumably because this uses just under
$16$ XMM registers.

\textbf{Testing AMD MagnyCours Pipelining} We also tried many possible
re-orderings of the various loads, adds, and multiplies involved in making the
$4\times 4$ updates. This made no measurable difference in runtimes, presumably
because the compiler automates this (we found out about this when spending time
in selecting compile flags).

\textbf{Testing Inner Product Methods} We originally implemented an
inner-product-based method, but were stuck on how to efficiently compute
reductions using SSE instructions (without SSE3). One major advantage of the
outer-product approach is that we do not have to resort to reduce operations.

\subsection{Memory Layout}
\label{sec:mem-layout}
One option in computing $C$ according to Equation~\eqref{eq:matmul} involves
(a) iterating over $i$, (b) iterating over $j$, and (c) in an inside loop,
computing $A_i B^{(j)}$. However, this does not take advantage of blocking
strategies. Instead, we make the $4\times 4$ updates as we iterate
over blocks of $A$ and rows of $B$ instead. We now describe how we lay out $A$,
$B$, and $C$, keeping this iteration strategy in mind. In general, all auxiliary
matrices are assigned on the stack, unless they are larger than
$1024 \times 1024$, in which case they are made on the heap.

Because we iterate over blocks of $A$, we store an
auxiliary matrix that stores entries of A columnwise block-by-block where each
block also has entries in column-major order. We make 16-byte-aligned memory
allocations on the stack, (a) to prepare for aligned SSE instructions and (b)
because matrix sizes tested are under 10 MB in size. We found that $68\times 68$
block sizes yielded good performance in practice. This procedure involves $4$
nested loops -- we tried all $24$ possible configurations and retained the best
ordering (see code for specifics).

Because we iterate over the rows of $B$, in processing
each block of $A$, we (a) copy the transpose of the block considered in $B$ into
an auxiliary matrix stored on the stack. Given $68\times 68$ blocks of $A$ and
$B$, we invoke the method described in Section~\ref{sec:ioproducts}.

We store $C$ in standard, column-major format. Recall that $A$, $B$, and $C$
are all padded to ensure that the side length is a multiple of $4$. In practice,
we could have also gone for a blocked implementation of $C$, but did not find
this necessary to get high performance (maybe blocking here would yield even
higher performance though).

\textbf{Testing Exhaustive Block Sizes.} We use one level
of blocking so as to improve the chances of matrix entries entering either the
L1 or the L2 cache. As we see later in the report, the lack of multiple layers
of blocking leads to decreased performance with increased matrix sizes, since
large matrices cannot fully fit into cached memory.

For each matrix size, we exhaustively test all block sizes with multiple $4$ up
to the matrix size. Rather than testing this via a \texttt{for}-loop over block sizes
directly within \texttt{benchmark.c}, we wrote a bash script which (1) replaced the
\texttt{\#define BLOCK\_SIZE} line with the appropriate block size. This was vital,
since directly coding this loop in C without generating the code would yield
substantially slower results (by 4\%); this is likely because in this latter case,
the block size would not be a compile-time constant. For robustness, we ran each
(block size, matrix size) tuple a total of four times.

We summarize the results in Figure TODO. Per matrix size, there is not a block size
that wins unequivocally -- rather, there is often a range of block sizes that typically
attains similar performance. Ultimately, we use the following ranges of block sizes:
for \texttt{lda}$\in [1, 32]$, we use a block size of $32$. For \texttt{lda}$\in [33, 97]$,
we use a block size of $52$. For \texttt{lda}$\in [98, 192]$, we use a block size of $68$.
For \texttt{lda}$\in [193, 639]$, we use a block size of $88$. For \texttt{lda}$\in [640, \infty]$,
we use a block size of $80$.

To ensure that \texttt{BLOCK\_SIZE} is still a compile-time constant, we employ
code generation; in particular, we define an auxiliary \texttt{square\_dgemm\_BLCK} that
has been defined in a pre-processor macro and contains most of the logic, assuming fixed block
size \texttt{BLCK}. The preprocessor macro allows us to then elegantly state the various
instances of the method (this simulates the concept of static-templating in C++).

\subsection{Compiler Options}
We employ the following compiler options:
\begin{itemize}
  \item \texttt{-O2}: supported options are \texttt{-O1}, \texttt{-O2},
    \texttt{-O3}, \texttt{-O4}, and \texttt{-Ofast}.
    We went with \texttt{-O2}, which was slightly better than \texttt{-Ofast}.
  \item \texttt{-march=native}: this option adds in further options that are
    specific to the native machine.
  \item \texttt{-funroll-loops}: this unrolls loops whose sizes are known at
    compile time.
  \item \texttt{-fsched-pressure}: this attempts to (1) eliminate execution
    stalls in the CPU pipeline when data is unavailable and (2) enables register
    pressure sensitive instruction scheduling before allocating registers.
    Ideally, this option minimizes register spillover.
  \item \texttt{-fsched-spec-load-dangerous}: to be honest, this flag sounded
    cool, so we decided to include it. It seemed to eke out a tenth of a percent
    performance, so why not include it. This allows for speculative motions of
    more load instructions.
\end{itemize}

We also experimented with \texttt{-funroll-all-loops} (slowed down performance in
comparison with \texttt{-funroll-loops}), \texttt{-ffast-math} (surprisingly
did not improve performance), \texttt{-fdata-sections} (slowed down performance),
and \texttt{-fbranch-target-load-optimize2} (made no difference).

We also experimented with \texttt{-fprofile-arcs}; after enabling this
flag and adding \texttt{-lgcov} to LDLIBS, we run our executable, which saves a
``.gcda'' file to disk. This file contains the various branch prediction
probabilities and other useful goodies, which we can then be used by future
executions of the code. Indeed, recompiling with the
\texttt{-fbranch-probabilities} flag, we unfortunately found that the resulting
code somehow ran slower (perhaps we didn't run something correctly).

\section{Testing on an Intel i7-4930K CPU}
We additionally tested our code on an Intel i7-4930K CPU to see how performance would
be impacted, given that we heavily optimized our code and compile flags particularly
for the Hopper nodes. We did have to make a slight modification to our code; discussed
in Section~\ref{sec:mem-layout}, we allocate matrices smaller than $1024 \times 1024$
on the stack -- we had to decrease this size to $639\times 639$ to avoid stack overflows.
In order to run the provided BLAS benchmark, we had to link via \texttt{-lblas}. Other
than these two changes, we did not modify the code or other flags.

Shown in Figure, our code outperforms the native BLAS by a large factor.
In particular, the BLAS libraries were not installed from source, so there was no
opportunity for the compiler to specifically optimize the code to this machine.
However, our code was compiled on this machine, allowing the gcc compiler to specifically
optimize performance.

\small
\bibliographystyle{plain}
\bibliography{main}

\end{document}

